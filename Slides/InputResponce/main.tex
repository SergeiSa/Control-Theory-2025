\documentclass{beamer}

\input{settings.tex}




\title{Input Response}
\subtitle{Control Theory, Lecture 6}
\author{by Sergei Savin}
\centering
\date{\mydate}



\begin{document}
\maketitle


%\begin{frame}{Content}
%
%\begin{itemize}
%\item Frequency response
%\item Partial-fraction expansion with sine input
%\item Amplitude and phase shift of a steady-state solution
%\item Bode plot
%\end{itemize}
%
%\end{frame}




\begin{frame}{Steady-State gain: ODE}
	% \framesubtitle{O}
	\begin{flushleft}
		
		Given an ODE with a constant input $c = \text{const}$:
		
		\begin{align}
			&a_n y^{(n)} + ... + a_1 \dot y  + a_0 y = b_m u^{(m )}+ ... + b_1 \dot u + b_0 u
			\\
			&u(t) = c 
		\end{align}
		
		This is equivalent to:
		
		\begin{align}
			a_n y^{(n)} + ... + a_1 \dot y  + a_0 y =  b_0 c
		\end{align}
		
		A steady-state solution $y_{ss} = \text{const}$:
		
		\begin{align}
			a_0 y_{ss}  =  b_0 c \\
			y_{ss} = \frac{b_0}{a_0} c
		\end{align}
		
		The quantity $K = \frac{b_0}{a_0}$ is the steady-state gain of the system.
		
	\end{flushleft}
\end{frame}





\begin{frame}{Steady-State gain: Transfer Function}
	%	\framesubtitle{Interesting things done easy}
	\begin{flushleft}
		
		Assume the system  $\mathcal{G}$ represented as a transfer function:
		
		\begin{equation}
			Y(s) = \frac{b_m s^m + ... + b_0}{a_n s^n + ... + a_0} U(s)
		\end{equation}
		
		\bigskip
		
		Then, as any element multiplied by the differential operator $s$ with power higher than 0 is a derivative of $u$ or $y$ and both are 0 at the steady-state solution, the steady-state gain can be found by setting those to zero:
		
		\begin{equation}
			K = \frac{b_0}{a_0}
		\end{equation}		

\end{flushleft}
\end{frame}		



\begin{frame}{Steady-State gain: State Space, 1}
	% \framesubtitle{O}
	\begin{flushleft}
		
		Given an LTI with a constant input $\bo{u}_{ss} = \text{const}$:
		
		\begin{equation}
			\begin{cases}
				\dot{\bo{x}} = \bo{A}\bo{x} + \bo{B}\bo{u}
				\\
				\bo{y} = \bo{C}\bo{x}
			\end{cases}
		\end{equation}
		
		A steady-state solution $\bo{x}_{ss} = \text{const}$:
		
		\begin{equation}
		\begin{cases}
			0 = \bo{A}\bo{x}_{ss} + \bo{B}\bo{u}_{ss}
			\\
			\bo{y}_{ss} = \bo{C}\bo{x}_{ss} 
		\end{cases}
		\end{equation}
		
		\begin{equation}
				\bo{y}_{ss} = -\bo{C} \bo{A}^{-1}\bo{B}\bo{u}_{ss}
		\end{equation}
		
		
		The quantity $K =-\bo{C} \bo{A}^{-1}\bo{B}$ is the steady-state gain of the system.
		
	\end{flushleft}
\end{frame}



\begin{frame}{Steady-State gain: State Space, 2}
	% \framesubtitle{O}
	\begin{flushleft}
		
		For the following LTI:
		
		\begin{equation}
			\begin{cases}
			\begin{bmatrix}
				\dot{x}_1 \\ \dot{x}_2 \\ \dot{x}_3
			\end{bmatrix} 
			=
			\begin{bmatrix}
				0 & 1 & 0 \\ 
				0 & 0 & 1 \\
				-a_0 & -a_1 & -a_2
			\end{bmatrix} 
			\begin{bmatrix}
				x_1 \\ x_2 \\ x_3
			\end{bmatrix} 
			+ 
			\begin{bmatrix}
				0 \\ 0 \\ b_0
			\end{bmatrix}
			\bo{u}
			\\
			\bo{y} = 			
			\begin{bmatrix}
				1 & 0 & 0
			\end{bmatrix}
			\begin{bmatrix}
			x_1 \\ x_2 \\ x_3
			\end{bmatrix} 
		\end{cases}
		\end{equation}
		%
		Then we get:
		%
		\begin{equation}
			\bo{y}_{ss} = 
			-\begin{bmatrix}
				1 & 0 & 0
			\end{bmatrix}
			\begin{bmatrix}
				-a_1 / a_0 & -a_2 / a_0 & -1 / a_0 \\
				1 & 0 & 0 \\
				0 & 1 & 0
			\end{bmatrix}
			\begin{bmatrix}
				0 \\ 0 \\ b_0
			\end{bmatrix}
			\bo{u}_{ss}
		\end{equation}
		
		
		The quantity $K =-\bo{C} \bo{A}^{-1}\bo{B} = \frac{b_0}{a_0}$ is the steady-state gain of the system.
		
	\end{flushleft}
\end{frame}






\begin{frame}{Frequency Response, 1}
	% \framesubtitle{O}
	\begin{flushleft}
		
		Consider LTI with input $\bo{u} = \alpha \cos(\omega t) + \beta \sin (\omega t)$:
		%
		\begin{equation}
			\begin{cases}
				\dot{\bo{x}} = \bo{A}\bo{x} + \bo{B}\bo{u}
				\\
				\bo{y} = \bo{C}\bo{x}
			\end{cases}
		\end{equation}
		%
		with steady-state solution:
		%
		\begin{align}
			x_i &= g_i \cos(\omega t) + h_i \sin (\omega t)
			\\
			\dot x_i &=\omega h_i \cos(\omega t) - \omega g_i \sin (\omega t)
			\\
			y_i &= q \cos(\omega t) + r \sin (\omega t)
		\end{align}
		%
		Grouping terms in front of cosines we get:
		%
		\begin{align}
			\begin{cases}
				\omega \bo{h} = \bo{A} \bo{g} + \bo{B} \alpha
				\\
				q = \bo{C} \bo{g}
			\end{cases}
		\end{align}
		%
		And grouping terms in front of sines :
		%
		\begin{align}
			\begin{cases}
				-\omega \bo{g} = \bo{A} \bo{h} + \bo{B} \beta
				\\
				r = \bo{C} \bo{h}
			\end{cases}
		\end{align}
		
		
	\end{flushleft}
\end{frame}




\begin{frame}{Frequency Response, 2}
	% \framesubtitle{O}
	\begin{flushleft}
		
		This system can be written as:
		
		\begin{align}
			\begin{cases}
				\begin{bmatrix}
					-\bo{A} & \omega \bo{I} \\
					-\omega \bo{I} & -\bo{A}
				\end{bmatrix}
				\begin{bmatrix}
					\bo{g} \\ \bo{h}
				\end{bmatrix}
				=
				\begin{bmatrix}
					\bo{B} & 0 \\  0 & \bo{B}
				\end{bmatrix}
				\begin{bmatrix}
				\alpha \\   \beta
				\end{bmatrix}
				\\
				\begin{bmatrix}
					q \\  r
				\end{bmatrix}
				= 
				\begin{bmatrix}
					\bo{C} & 0 \\  0 & \bo{C}
				\end{bmatrix}
				\begin{bmatrix}
				\bo{g} \\ \bo{h}
				\end{bmatrix}
			\end{cases}
		\end{align}		
		
		Note that we can compute the steady-state solution for all states of this system:
		%
		\begin{align}
			\begin{bmatrix}
				\bo{g} \\ \bo{h}
			\end{bmatrix}
			= 
				\begin{bmatrix}
					-\bo{A} & \omega \bo{I} \\
					-\omega \bo{I} & -\bo{A}
				\end{bmatrix}^{-1}
				\begin{bmatrix}
					\bo{B} & 0 \\  0 & \bo{B}
				\end{bmatrix}
				\begin{bmatrix}
					\alpha \\   \beta
				\end{bmatrix}
		\end{align}		
		
		We can also solve for the steady-state output:
		%
		\begin{align}
			\begin{bmatrix}
				q \\  r
			\end{bmatrix}
			= 
			\begin{bmatrix}
				\bo{C} & 0 \\  0 & \bo{C}
			\end{bmatrix}
			\begin{bmatrix}
				-\bo{A} & \omega \bo{I} \\
				-\omega \bo{I} & -\bo{A}
			\end{bmatrix}^{-1}
			\begin{bmatrix}
				\bo{B} & 0 \\  0 & \bo{B}
			\end{bmatrix}
			\begin{bmatrix}
				\alpha \\   \beta
			\end{bmatrix}
		\end{align}		
		
		
	\end{flushleft}
\end{frame}






\begin{frame}{Frequency Response, 3}
	% \framesubtitle{O}
	\begin{flushleft}
		
		The map between the input and the output as:
		%
		\begin{align}
			\bo{M}(\omega)
			= 
			\begin{bmatrix}
				\bo{C} & 0 \\  0 & \bo{C}
			\end{bmatrix}
			\begin{bmatrix}
				-\bo{A} & \omega \bo{I} \\
				-\omega \bo{I} & -\bo{A}
			\end{bmatrix}^{-1}
			\begin{bmatrix}
				\bo{B} & 0 \\  0 & \bo{B}
			\end{bmatrix}
		\end{align}		
		
		We can define input coordinates $\zeta = 
		\begin{bmatrix}
			\alpha \\   \beta
		\end{bmatrix}$ and output coordinates $\bo{p} = 
		\begin{bmatrix}
		q \\  r
		\end{bmatrix} = \bo{M}\zeta$. The amptitude amplification can be defined as the ratio:
		%
		\begin{align}
			\text{amp}(\omega)
			= 
			\frac{\sqrt{q^2 + r^2}}{\sqrt{\alpha^2 + \beta^2}}
			=
			\frac{||\bo{p}||}{||\zeta||}
			=
			\frac{||\bo{M}\zeta||}{||\zeta||}
		\end{align}		
		
		Notice that by definition $\underset{\zeta}{\text{max}}\frac{||\bo{M}\zeta||_2}{||\zeta||_2} = ||\bo{M}||_2 = \sigma_{\max}(\bo{M})$, where $\sigma_{\max}(\bo{M})$ is the largest singular value of the matrix.
		
	\end{flushleft}
\end{frame}




\begin{frame}{Frequency Response, 4}
	% \framesubtitle{O}
	\begin{flushleft}
		
		The phase shift can be defined as:
		%
		\begin{align}
			\text{phase}(\omega)
			\sim 
			\angle (\bo{p}) - \angle (\zeta)
			=
			\angle (\bo{M} \zeta) - \angle (\zeta)
		\end{align}		
		
		But matrix $\bo{M}$ is an orthogonal matrix:  $\bo{M} = \text{amp}(\omega)
		\begin{bmatrix}
			\cos(\varphi) & -\sin(\varphi) \\
			\sin(\varphi) & \cos(\varphi)
		\end{bmatrix}$. Thus:
		%
		\begin{align}
			\text{phase}(\omega)
			\sim 
			\varphi = \text{atan2}(M_{21}, M_{11})
		\end{align}		
		
	\end{flushleft}
\end{frame}



\begin{frame}{MIMO Frequency Response, 1}
	% \framesubtitle{O}
	\begin{flushleft}
		
		Transfer function-based Bode plot relies on a SISO representation of a system. However, choosing input and output one can use it for a MIMO system as well. 
		
		\bigskip
		
		State-space representation naturally points out the connection between inputs and outputs and the resulting responce. Consider a fixed input matrix $\bo{B} \in \R^{n \times 1}$; we can ask a question, what choice of the output $\bo{C} \in \R^{1 \times n}$ (assuming $||\bo{C} || = 1$) produces the largest amplitude of the output signal.
		
	\end{flushleft}
\end{frame}




\begin{frame}{MIMO Frequency Response, 2}
	% \framesubtitle{O}
	\begin{flushleft}
		
		Fropm the previous slides we saw that:
		
		\begin{align*}
			 \text{amp}(\omega)
			\begin{bmatrix}
				\cos(\varphi) & -\sin(\varphi) \\
				\sin(\varphi) & \cos(\varphi)
			\end{bmatrix}
			= 
			\begin{bmatrix}
				\bo{C} & 0 \\  0 & \bo{C}
			\end{bmatrix}
			\begin{bmatrix}
				-\bo{A} & \omega \bo{I} \\
				-\omega \bo{I} & -\bo{A}
			\end{bmatrix}^{-1}
			\begin{bmatrix}
				\bo{B} & 0 \\  0 & \bo{B}
			\end{bmatrix}
		\end{align*}		
		
		Defining:
		%
		\begin{align}
			\begin{bmatrix}
				\bo{P} _{11} & \bo{P} _{12} \\
				\bo{P} _{21} & \bo{P} _{22}
			\end{bmatrix}
			= 
			\begin{bmatrix}
				-\bo{A} & \omega \bo{I} \\
				-\omega \bo{I} & -\bo{A}
			\end{bmatrix}^{-1}
			\begin{bmatrix}
				\bo{B} & 0 \\  0 & \bo{B}
			\end{bmatrix},
			\ \ \
			\text{amp}(\omega) = m
		\end{align}		
		
		We find:
		%
		\begin{align}
			\begin{bmatrix}
				m\cos(\varphi) & -m\sin(\varphi) \\
				m\sin(\varphi) & m\cos(\varphi)
			\end{bmatrix}
			= 
			\begin{bmatrix}
			\bo{C}\bo{P} _{11} & \bo{C}\bo{P} _{12} \\
			\bo{C}\bo{P} _{21} & \bo{C}\bo{P} _{22}
			\end{bmatrix}
			\\
			m^2\cos^2(\varphi) + m^2\sin^2(\varphi)
		= 
		\bo{C}\bo{P} _{11} \bo{P} _{11}\T \bo{C}\T + 
		\bo{C}\bo{P} _{21} \bo{P} _{21}\T \bo{C}\T
		\\
		m^2
		= 
		\bo{C}(\bo{P} _{11} \bo{P} _{11}\T +\bo{P} _{21} \bo{P} _{21}\T  )\bo{C}\T
		\end{align}		
		
		
		
	\end{flushleft}
\end{frame}



\begin{frame}{MIMO Frequency Response, 3}
	% \framesubtitle{O}
	\begin{flushleft}
		
		Defining $\bo{N} = \bo{P} _{11} \bo{P} _{11}\T +\bo{P} _{21} \bo{P} _{21}\T$ with decomposition $\bo{N} = \bo{D} \bo{D}\T$ we can find maximum value of $m$ by maximizing:
		%
		\begin{align*}
			m = \underset{\bo{C}}{\max } \frac{\sqrt{\bo{C} \bo{D} \bo{D}\T \bo{C}\T}}{|| \bo{C}||}
			=
			\underset{\bo{C}}{\max } \frac{||\bo{D}\T \bo{C}\T||}{|| \bo{C}||} = \sigma_{\max}(\bo{D}) =\sqrt{ \sigma_{\max}(\bo{N})}
		\end{align*}		
		
		This allows us:
		
		\begin{itemize}
			\item To find highest amplification ration in the system's state-space.
			
			\item To determine the output matrix corresponding to this "worse-case scenario"  $\bo{C}_{\max}$ as the vector in the SVD decomposition of $\bo{N})$ matrix corresponding to the largest singular value.
		\end{itemize}
		
	\end{flushleft}
\end{frame}



\begin{frame}{MIMO Frequency Response, 4}
	% \framesubtitle{O}
	\begin{flushleft}
		
		% TODO: \usepackage{graphicx} required
		\begin{figure}
			\centering
			\includegraphics[width=1 \linewidth]{WorstCaseAmp}
			\caption{Dashed - worst case amplitude response, other two - a particular bode plot}
			\label{fig:worstcaseamp}
		\end{figure}
		
		
	\end{flushleft}
\end{frame}


\myqrframe






\end{document}
