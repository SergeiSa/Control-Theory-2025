\documentclass{beamer}

\pdfmapfile{+sansmathaccent.map}


\mode<presentation>
{
	\usetheme{Warsaw} % or try Darmstadt, Madrid, Warsaw, Rochester, CambridgeUS, ...
	\usecolortheme{seahorse} % or try seahorse, beaver, crane, wolverine, ...
	\usefonttheme{serif}  % or try serif, structurebold, ...
	\setbeamertemplate{navigation symbols}{}
	\setbeamertemplate{caption}[numbered]
} 


%%%%%%%%%%%%%%%%%%%%%%%%%%%%
% itemize settings


%%%%%%%%%%%%%%%%%%%%%%%%%%%%
% itemize settings

\definecolor{myhotpink}{RGB}{255, 80, 200}
\definecolor{mywarmpink}{RGB}{255, 60, 160}
\definecolor{mylightpink}{RGB}{255, 80, 200}
\definecolor{mypink}{RGB}{255, 30, 80}
\definecolor{mydarkpink}{RGB}{155, 25, 60}

\definecolor{mypaleblue}{RGB}{240, 240, 255}
\definecolor{mylightblue}{RGB}{120, 150, 255}
\definecolor{myblue}{RGB}{90, 90, 255}
\definecolor{mygblue}{RGB}{70, 110, 240}
\definecolor{mydarkblue}{RGB}{0, 0, 180}
\definecolor{myblackblue}{RGB}{40, 40, 120}

\definecolor{mygreen}{RGB}{0, 200, 0}
\definecolor{mygreen2}{RGB}{245, 255, 230}
\definecolor{mydarkgreen}{RGB}{0, 120, 0}


\definecolor{mygray}{gray}{0.8}
\definecolor{mydarkgray}{RGB}{80, 80, 160}
\definecolor{mylightgray}{RGB}{160, 160, 160}

\definecolor{mydarkred}{RGB}{160, 30, 30}
\definecolor{mylightred}{RGB}{255, 150, 150}
\definecolor{myred}{RGB}{200, 110, 110}
\definecolor{myblackred}{RGB}{120, 40, 40}

\definecolor{mypink}{RGB}{255, 30, 80}
\definecolor{myhotpink}{RGB}{255, 80, 200}
\definecolor{mywarmpink}{RGB}{255, 60, 160}
\definecolor{mylightpink}{RGB}{255, 80, 200}
\definecolor{mydarkpink}{RGB}{155, 25, 60}
\definecolor{mywhitepink}{RGB}{255, 240, 240}

\definecolor{mydarkcolor}{RGB}{60, 25, 155}
\definecolor{mylightcolor}{RGB}{130, 180, 250}

\setbeamertemplate{itemize items}[default]

\setbeamertemplate{itemize item}{\color{myblackblue}$\blacksquare$}
\setbeamertemplate{itemize subitem}{\color{mygblue}$\blacktriangleright$}
\setbeamertemplate{itemize subsubitem}{\color{mygray}$\blacksquare$}

\setbeamercolor{palette quaternary}{fg=white,bg=mydarkgray}
\setbeamercolor{titlelike}{parent=palette quaternary}

\setbeamercolor{palette quaternary2}{fg=black,bg=mypaleblue}
\setbeamercolor{frametitle}{parent=palette quaternary2}

\setbeamerfont{frametitle}{size=\Large,series=\scshape}
\setbeamerfont{framesubtitle}{size=\normalsize,series=\upshape}





%%%%%%%%%%%%%%%%%%%%%%%%%%%%
% block settings

\setbeamercolor{block title}{bg=red!30,fg=black}

\setbeamercolor*{block title example}{bg=mygreen!40!white,fg=black}

\setbeamercolor*{block body example}{fg= black, bg= mygreen2}


%%%%%%%%%%%%%%%%%%%%%%%%%%%%
% URL settings
\hypersetup{
	colorlinks=true,
	linkcolor=blue,
	filecolor=blue,      
	urlcolor=blue,
}

%%%%%%%%%%%%%%%%%%%%%%%%%%

\renewcommand{\familydefault}{\rmdefault}

\usepackage{amsmath}
\usepackage{mathtools}

\usepackage{subcaption}

\usepackage{qrcode}

\newcommand{\bo}[1] {\mathbf{#1}}
\newcommand{\R}{\mathbb{R}} 
\newcommand{\T}{^\top}     



\newcommand{\mydate}{Spring 2025}

\newcommand{\mygit}{\textcolor{blue}{\href{https://github.com/SergeiSa/Control-Theory-2024}{github.com/SergeiSa/Control-Theory-2024}}}

\newcommand{\myqr}{ \textcolor{black}{\qrcode[height=1.5in]{https://github.com/SergeiSa/Control-Theory-2024}}
}

\newcommand{\myqrframe}{
	\begin{frame}
		\centerline{Lecture slides are available via Github:}
		\bigskip
		\centerline{\mygit}
		\bigskip
		\myqr
	\end{frame}
}


\newcommand{\bref}[2] {\textcolor{blue}{\href{#1}{#2}}}

%%%%%%%%%%%%%%%%%%%%%%%%%%%%
% code settings

\usepackage{listings}
\usepackage{color}
% \definecolor{mygreen}{rgb}{0,0.6,0}
% \definecolor{mygray}{rgb}{0.5,0.5,0.5}
\definecolor{mymauve}{rgb}{0.58,0,0.82}
\lstset{ 
	backgroundcolor=\color{white},   % choose the background color; you must add \usepackage{color} or \usepackage{xcolor}; should come as last argument
	basicstyle=\footnotesize,        % the size of the fonts that are used for the code
	breakatwhitespace=false,         % sets if automatic breaks should only happen at whitespace
	breaklines=true,                 % sets automatic line breaking
	captionpos=b,                    % sets the caption-position to bottom
	commentstyle=\color{mygreen},    % comment style
	deletekeywords={...},            % if you want to delete keywords from the given language
	escapeinside={\%*}{*)},          % if you want to add LaTeX within your code
	extendedchars=true,              % lets you use non-ASCII characters; for 8-bits encodings only, does not work with UTF-8
	firstnumber=0000,                % start line enumeration with line 0000
	frame=single,	                   % adds a frame around the code
	keepspaces=true,                 % keeps spaces in text, useful for keeping indentation of code (possibly needs columns=flexible)
	keywordstyle=\color{blue},       % keyword style
	language=Octave,                 % the language of the code
	morekeywords={*,...},            % if you want to add more keywords to the set
	numbers=left,                    % where to put the line-numbers; possible values are (none, left, right)
	numbersep=5pt,                   % how far the line-numbers are from the code
	numberstyle=\tiny\color{mygray}, % the style that is used for the line-numbers
	rulecolor=\color{black},         % if not set, the frame-color may be changed on line-breaks within not-black text (e.g. comments (green here))
	showspaces=false,                % show spaces everywhere adding particular underscores; it overrides 'showstringspaces'
	showstringspaces=false,          % underline spaces within strings only
	showtabs=false,                  % show tabs within strings adding particular underscores
	stepnumber=2,                    % the step between two line-numbers. If it's 1, each line will be numbered
	stringstyle=\color{mymauve},     % string literal style
	tabsize=2,	                   % sets default tabsize to 2 spaces
	title=\lstname                   % show the filename of files included with \lstinputlisting; also try caption instead of title
}


%%%%%%%%%%%%%%%%%%%%%%%%%%%%
% URL settings
\hypersetup{
	colorlinks=false,
	linkcolor=blue,
	filecolor=blue,      
	urlcolor=blue,
}

%%%%%%%%%%%%%%%%%%%%%%%%%%

%%%%%%%%%%%%%%%%%%%%%%%%%%%%
% tikz settings

\usepackage{tikz}
\tikzset{every picture/.style={line width=0.75pt}}


\title{LMI: Control design and robustness}
\subtitle{Control Theory, Lecture ??}
\author{by Sergei Savin}
\centering
\date{\mydate}



\begin{document}
	\maketitle
	
	
	
	\begin{frame}{Content}
		\begin{itemize}
			\item LMI
			\item Control design
			\item Robustness
			\item S-procedure
			\item Appendix A
		\end{itemize}
	\end{frame}
	
	
	
	
	\begin{frame}{linear matrix inequalities (LMI)}
		%\framesubtitle{How do we know the state?}
		\begin{flushleft}
			
			A linear matrix inequality (LMI) is a semidefinite constraint placed on a matrix:
			
			\begin{equation}
				\bo{S} \succ 0
			\end{equation}
			
			We assume (and this is true!) that there exist \emph{solvers} that can solve problems with such constraints. 
			
			
			\begin{example}
				Given $\bo{A}$, find such $\bo{S}\succ 0$ that $\bo{A}^\top\bo{S} + \bo{S}\bo{A} \prec 0$.
			\end{example}
			
			Notice that the last example is continious-time Lyapunov eq. for LTI system $\dot{\bo{x}} = \bo{A}\bo{x}$, and if such $\bo{S}$ exists the system is stable. 
			
		\end{flushleft}
	\end{frame}
	
	
	
	\begin{frame}{Control design, 1}
		%	\framesubtitle{Part 1}
		\begin{flushleft}
			
			Consider a system $\dot{\bo{x}} = \bo{A}\bo{x} + \bo{B}\bo{u}$, control $\bo{u} = \bo{K}\bo{x}$ and a Lyapunov function $V = \bo{x}^\top\bo{S}\bo{x}$, $\bo{S} \succ 0$.
			
			\bigskip
			
			Closed-form of the system is $\dot{\bo{x}} = (\bo{A} + \bo{B}\bo{K})\bo{x}$, and full derivative of the Lyapunov function:
			
			\begin{equation}
				\dot V = \bo{x}^\top (\bo{A} + \bo{B}\bo{K})^\top\bo{S}\bo{x} + \bo{x}^\top\bo{S} (\bo{A} + \bo{B}\bo{K}) \bo{x} \leq 0
			\end{equation}
			
			This can be re-written as an LMI:
			
			\begin{equation}
				\label{eq:vdot}
				(\bo{A} + \bo{B}\bo{K})^\top\bo{S} + \bo{S} (\bo{A} + \bo{B}\bo{K}) \prec 0
			\end{equation}
			
			This is \emph{not linear} in decision variables ($\bo{S}$ and $\bo{K}$), and can't be solved directly using popular solvers.
			
		\end{flushleft}
	\end{frame}
	
	
	
	
	\begin{frame}{Control design, 2}
		%	\framesubtitle{Part 2}
		\begin{flushleft}
			
			Introducing new variable $\bo{P} = \bo{S}^{-1}$ and multiplying \eqref{eq:vdot} by $\bo{P}$ on both sides (we can do it, as both $\bo{P}$ and $\bo{S}$ are full rank, and thus it is a congruence transformation which preserves definiteness, see appendix) we get:
			
			\begin{equation}
				\bo{P}(\bo{A} + \bo{B}\bo{K})^\top + (\bo{A} + \bo{B}\bo{K})\bo{P} \prec 0
			\end{equation}
			
			Now we introduce one more variable $\bo{L} = \bo{K}\bo{P}$ and get an LMI constraint:
			
			\begin{equation}
				\label{control_design}
				\bo{P}\bo{A}^\top + \bo{A}\bo{P} + \bo{L}^\top\bo{B}^\top + \bo{B}\bo{L} \prec 0
			\end{equation}
			
			Solving \eqref{control_design} gives us $\bo{P}$ and $\bo{L}$, from which we can compute $\bo{K} = \bo{L}\bo{P}^{-1}$ and $\bo{S} = \bo{P}^{-1}$, solving the original problem.
			
		\end{flushleft}
	\end{frame}
	
	
	
	
	\begin{frame}{Robustness, 1}
		%	\framesubtitle{Part 1}
		\begin{flushleft}
			
			Consider a system $\dot{\bo{x}} = \bo{A}\bo{x}$, but when you don't know $\bo{A}$ exactly. In other words, you don't know the model exactly. This is not to say that we know nothing about the model, but there is an uncertainty in our knowledge.
			
			\bigskip
			
			A good way to model is luck of model knowledge, this \emph{uncertainty}, is this:
			
			\begin{equation}
				\label{eq:uncertain}
				\dot{\bo{x}} = (\bo{A} + \bo{F} \Delta \bo{E})\bo{x}
			\end{equation}
			%
			where $\bo{F}$ and $\bo{E}$ are arbitrary matrices, and $\Delta$ is a \emph{norm-bounded} matrix: $||\Delta|| \leq 1$.
			
			\bigskip
			
			We can think of it this way: $\bo{A} + \bo{F} \Delta \bo{E}$ is the true but unknown model, and the range of all possible models we can expect is bounded by the possible values of $\Delta$.
			
		\end{flushleft}
	\end{frame}
	
	
	
	\begin{frame}{Robustness, 2}
		%	\framesubtitle{Part 1}
		\begin{flushleft}
			
			Lets write the Lyapunov equation for the system \eqref{eq:uncertain}:
			
			\begin{equation}
				\dot V = \bo{x}^\top 
				(\bo{A} + \bo{F} \Delta \bo{E})^\top\bo{S}\bo{x} + \bo{x}^\top\bo{S} (\bo{A} + \bo{F} \Delta \bo{E}) \bo{x} \leq 0
			\end{equation}
			
			Let us introduce a new variable $\bo{w} = \Delta \bo{E}\bo{x}$:
			
			\begin{equation}
				\label{eq:Lyapunov_xw}
				\dot V = \bo{x}^\top 
				(\bo{A}^\top \bo{S} + \bo{S}\bo{A}) \bo{x} + 
				\bo{w}^\top \bo{F}^\top\bo{S} \bo{x} +
				\bo{x}^\top \bo{S}\bo{F} \bo{w} \leq 0
			\end{equation}
			
			Let us consider $\bo{w}^\top \bo{w}$:
			
			\begin{equation}
				\label{eq:Delta-inequality}
				\bo{w}^\top \bo{w} = 
				\bo{x}^\top\bo{E}^\top \Delta \Delta \bo{E}\bo{x}
				\leq
				\bo{x}^\top\bo{E}^\top \bo{E}\bo{x}
			\end{equation}		
			%
			which is true because $|| \Delta ||\leq 1$. In fact, the only property of the norm that we need here is that the delta inequality \eqref{eq:Delta-inequality} holds.
			
		\end{flushleft}
	\end{frame}
	
	
	
	\begin{frame}{Robustness, 3}
		%	\framesubtitle{Part 1}
		\begin{flushleft}
			
			With $\bo{w}^\top \bo{w} 
			\leq
			\bo{x}^\top\bo{E}^\top \bo{E}\bo{x}$ we can write:
			
			\begin{equation}
				\bo{x}^\top\bo{E}^\top \bo{E}\bo{x} - \bo{w}^\top \bo{w} \geq 0
			\end{equation}	
			
			Which is the same as:
			
			\begin{equation}
				\label{eq:EEww}
				\begin{bmatrix}
					\bo{x} \\ \bo{w}
				\end{bmatrix}^\top
				\begin{bmatrix}
					\bo{E}^\top \bo{E} & 0 \\
					0 & -\bo{I}
				\end{bmatrix}		
				\begin{bmatrix}
					\bo{x} \\ \bo{w}
				\end{bmatrix}
				\geq 0
			\end{equation}	
			
			The same way we can rewrite \eqref{eq:Lyapunov_xw}:
			
			\begin{equation}
				\label{eq:xw_Lyapunov}
				\begin{bmatrix}
					\bo{x} \\ \bo{w}
				\end{bmatrix}^\top
				\begin{bmatrix}
					\bo{A}^\top \bo{S} + \bo{S}\bo{A} & \bo{S}\bo{F} \\
					\bo{F}^\top\bo{S} & 0
				\end{bmatrix}		
				\begin{bmatrix}
					\bo{x} \\ \bo{w}
				\end{bmatrix}
				\leq 0
			\end{equation}	
			%
			which only need to hold while \eqref{eq:EEww} holds.
			
		\end{flushleft}
	\end{frame}
	
	
	
	\begin{frame}{S-procedure}
		%	\framesubtitle{Part 1}
		\begin{flushleft}
			
			There is a way to enforce constraint $\bo{z}^\top \bo{M}\bo{z} \leq 0$ for such $\bo{z}$ that $\bo{z}^\top \bo{N}\bo{z} \geq 0$. This is called \emph{s-procedure}.
			
			\begin{theorem}
				If $\gamma > 0$ and $\bo{M} + \gamma \bo{N} \prec 0$ then $\bo{z}^\top \bo{N}\bo{z} \geq 0 \implies \bo{z}^\top \bo{M}\bo{z} \leq 0$ 
			\end{theorem}
			
		\end{flushleft}
	\end{frame}
	
	
	
	\begin{frame}{Robustness, 4}
		%	\framesubtitle{Part 1}
		\begin{flushleft}
			
			
			Using s-procedure we enforce \eqref{eq:xw_Lyapunov} when \eqref{eq:EEww} holds:
			
			\begin{equation}
				%			\label{eq:EEww}
				\begin{bmatrix}
					\bo{x} \\ \bo{w}
				\end{bmatrix}^\top
				\begin{bmatrix}
					\bo{A}^\top \bo{S} + \bo{S}\bo{A} + \gamma \bo{E}^\top \bo{E} & \bo{S}\bo{F} \\
					\bo{F}^\top\bo{S} & -\gamma\bo{I}
				\end{bmatrix}		
				\begin{bmatrix}
					\bo{x} \\ \bo{w}
				\end{bmatrix}
				\leq 0
			\end{equation}
			
			In LMI form this is:
			
			\begin{equation}
				\begin{bmatrix}
					\bo{A}^\top \bo{S} + \bo{S}\bo{A} + \gamma \bo{E}^\top \bo{E} & \bo{S}\bo{F} \\
					\bo{F}^\top\bo{S} & -\gamma\bo{I}
				\end{bmatrix}		
				\prec 0
			\end{equation}	
			
			
			This is a condition that the system is stable for all values of $\Delta$. The decision variables are $\bo{S}$ and $\gamma$.		
			
		\end{flushleft}
	\end{frame}
	
	
	
	
	
	\begin{frame}{Quadratic stability, 1}
		%	\framesubtitle{Part 1}
		\begin{flushleft}
			
			Let us consider the following system:
			
			\begin{equation}
				\dot{\bo{x}} = \bo{A}\bo{x}
			\end{equation}
			%
			where $\bo{A} = \sum\limits_{i=1}^{n} \alpha_i \bo{A}_i$, $\alpha_i \geq 0$, $\sum\limits_{i=1}^{n} \alpha_i = 1$ with known $\bo{A}_i$ but unknown coefficients $\alpha_i$. Is it stable for all possible values of $\alpha_i$? Note that we can't use eigenvalue analysis in this case.
			
			\bigskip
			
			Geometrically, this means $\bo{A}$ is in a polytope with vertices $\bo{A}_i$.
			
		\end{flushleft}
	\end{frame}
	
	
	
	\begin{frame}{Quadratic stability, 2}
		%	\framesubtitle{Part 1}
		\begin{flushleft}
			
			\begin{theorem}[Quadratic stability]
				$\bo{A}_i^\top \bo{S} + \bo{S} \bo{A}_i \leq 0$ implies $\dot{\bo{x}} = \sum\limits_{i=1}^{n} \alpha_i \bo{A}_i \bo{x}$ is stable, where $\alpha_i \geq 0$, $\sum\limits_{i=1}^{n} \alpha_i = 1$
			\end{theorem}
			
			\bigskip
			
			Proof: $\dot V = \left(\sum\limits_{i=1}^{n} \alpha_i \bo{A}_i \right)^\top \bo{S} + \bo{S} 
			\left( \sum\limits_{i=1}^{n} \alpha_i \bo{A}_i \right) \leq 0$ can be re-written as: 
			$\dot V = \sum\limits_{i=1}^{n} \left( \alpha_i (\bo{A}_i^\top \bo{S} + \bo{S} \bo{A}_i) \right) $ and since $\bo{A}_i^\top \bo{S} + \bo{S} \bo{A}_i \leq 0$ and $\alpha_i \geq 0$, then $\dot V \leq 0$. \qed
			
		\end{flushleft}
	\end{frame}
	
	
	
	
	\begin{frame}{Quadratic stability - Control design, 1}
		%	\framesubtitle{Part 1}
		\begin{flushleft}
			
			Let us consider the following system:
			
			\begin{equation}
				\dot{\bo{x}} = \bo{A}\bo{x} + \bo{B}\bo{x}
			\end{equation}
			%
			where $\bo{A} = \sum\limits_{i=1}^{n} \alpha_i \bo{A}_i$, $\alpha_i \geq 0$, $\sum\limits_{i=1}^{n} \alpha_i = 1$ with known $\bo{A}_i$ but unknown coefficients $\alpha_i$. How to design control law $\bo{u} = \bo{K}\bo{x}$ making the system stable for all possible values of $\alpha_i$? 
			
			\bigskip
			
			The closed-loop form of the system is:
			
			\begin{equation}
				\dot{\bo{x}} = (\sum\limits_{i=1}^{n} \alpha_i \bo{A}_i + \bo{B}\bo{K})\bo{x}
			\end{equation}
			
			
		\end{flushleft}
	\end{frame}
	
	
	
	\begin{frame}{Quadratic stability - Control design, 2}
		%	\framesubtitle{Part 1}
		\begin{flushleft}
			
			Let us write Lyapunov eq. for the system:
			
			\begin{equation}
				\left(
				\sum\limits_{i=1}^{n} \alpha_i (\bo{A}_i + \bo{B}\bo{K})
				\right)^\top \bo{S} 
				+ 
				\bo{S}
				\left(
				\sum\limits_{i=1}^{n} \alpha_i (\bo{A}_i + \bo{B}\bo{K})
				\right) 
				\prec 0
			\end{equation}
			
			We can re-write it as:
			
			\begin{equation}
				\sum\limits_{i=1}^{n} \alpha_i 
				\left( 
				(\bo{A}_i + \bo{B}\bo{K})^\top \bo{S} +
				\bo{S} (\bo{A}_i + \bo{B}\bo{K})
				\right)
				\prec 0
			\end{equation}
			
			Hence if $(\bo{A}_i + \bo{B}\bo{K})^\top \bo{S} +
			\bo{S} (\bo{A}_i + \bo{B}\bo{K}) \prec 0$, the original system is stable.
			
		\end{flushleft}
	\end{frame}
	
	
	
	\begin{frame}{Quadratic stability - Control design, 3}
		%	\framesubtitle{Part 1}
		\begin{flushleft}
			
			From $(\bo{A}_i + \bo{B}\bo{K})^\top \bo{S} +
			\bo{S} (\bo{A}_i + \bo{B}\bo{K}) \prec 0$, we can go on to do control design. Introducing $\bo{P} = \bo{S}^{-1}$, we use congruence transformation multiplying by $\bo{P}$ on both sides:
			
			\begin{equation}
				\bo{P}(\bo{A}_i + \bo{B}\bo{K})^\top  +
				(\bo{A}_i + \bo{B}\bo{K})\bo{P} \prec 0
			\end{equation}
			
			Introducing new variable $\bo{L} = \bo{K} \bo{P}$ we get a problem linear in decision variables:
			
			\begin{equation}
				\bo{P}\bo{A}_i^\top + \bo{A}_i \bo{P} +
				\bo{L}^\top\bo{B}^\top + \bo{B}\bo{L} \prec 0
			\end{equation}		
			%
			where the decision variables are $\bo{P}$ and $\bo{L}$. The control gain matrix is found as $\bo{K} = \bo{L} \bo{P}^{-1}$.
			
		\end{flushleft}
	\end{frame}
	
	
	
\myqrframe
	
	
	
	
	\begin{frame}{Appendix A}
		\framesubtitle{Congruence transformation and definiteness}
		\begin{flushleft}
			
			Consider matrices $\bo{P} \succ 0$, and $\bo{V} \in \R^{n, n}$ is full rank. We can prove that:
			
			\begin{equation}
				\bo{P} \succ 0 \implies \bo{V}^\top\bo{P}\bo{V} \succ 0
			\end{equation}
			
			Proof: $\bo{x}^\top\bo{V}^\top\bo{P}\bo{V}\bo{x} = \bo{z}^\top\bo{P}\bo{z}$, where $\bo{z} = \bo{V}\bo{x}$. Since $\bo{P} \succ 0$, $\bo{z}^\top\bo{P}\bo{z} \geq 0$, hence $\bo{x}^\top\bo{V}^\top\bo{P}\bo{V}\bo{x} \geq 0$. 
			
			\begin{definition}
				Congruence transformation preserves semi-definiteness: $\text{det}(\bo{V}) \neq 0, \ \bo{P} \succ 0 \implies \bo{V}^\top\bo{P}\bo{V} \succ 0$
			\end{definition}
			
			
		\end{flushleft}
	\end{frame}
	
	
	
	
\end{document}
