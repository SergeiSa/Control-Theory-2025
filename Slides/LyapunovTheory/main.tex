\documentclass{beamer}

\pdfmapfile{+sansmathaccent.map}


\mode<presentation>
{
	\usetheme{Warsaw} % or try Darmstadt, Madrid, Warsaw, Rochester, CambridgeUS, ...
	\usecolortheme{seahorse} % or try seahorse, beaver, crane, wolverine, ...
	\usefonttheme{serif}  % or try serif, structurebold, ...
	\setbeamertemplate{navigation symbols}{}
	\setbeamertemplate{caption}[numbered]
} 


%%%%%%%%%%%%%%%%%%%%%%%%%%%%
% itemize settings


%%%%%%%%%%%%%%%%%%%%%%%%%%%%
% itemize settings

\definecolor{myhotpink}{RGB}{255, 80, 200}
\definecolor{mywarmpink}{RGB}{255, 60, 160}
\definecolor{mylightpink}{RGB}{255, 80, 200}
\definecolor{mypink}{RGB}{255, 30, 80}
\definecolor{mydarkpink}{RGB}{155, 25, 60}

\definecolor{mypaleblue}{RGB}{240, 240, 255}
\definecolor{mylightblue}{RGB}{120, 150, 255}
\definecolor{myblue}{RGB}{90, 90, 255}
\definecolor{mygblue}{RGB}{70, 110, 240}
\definecolor{mydarkblue}{RGB}{0, 0, 180}
\definecolor{myblackblue}{RGB}{40, 40, 120}

\definecolor{mygreen}{RGB}{0, 200, 0}
\definecolor{mygreen2}{RGB}{245, 255, 230}
\definecolor{mydarkgreen}{RGB}{0, 120, 0}


\definecolor{mygray}{gray}{0.8}
\definecolor{mydarkgray}{RGB}{80, 80, 160}
\definecolor{mylightgray}{RGB}{160, 160, 160}

\definecolor{mydarkred}{RGB}{160, 30, 30}
\definecolor{mylightred}{RGB}{255, 150, 150}
\definecolor{myred}{RGB}{200, 110, 110}
\definecolor{myblackred}{RGB}{120, 40, 40}

\definecolor{mypink}{RGB}{255, 30, 80}
\definecolor{myhotpink}{RGB}{255, 80, 200}
\definecolor{mywarmpink}{RGB}{255, 60, 160}
\definecolor{mylightpink}{RGB}{255, 80, 200}
\definecolor{mydarkpink}{RGB}{155, 25, 60}
\definecolor{mywhitepink}{RGB}{255, 240, 240}

\definecolor{mydarkcolor}{RGB}{60, 25, 155}
\definecolor{mylightcolor}{RGB}{130, 180, 250}

\setbeamertemplate{itemize items}[default]

\setbeamertemplate{itemize item}{\color{myblackblue}$\blacksquare$}
\setbeamertemplate{itemize subitem}{\color{mygblue}$\blacktriangleright$}
\setbeamertemplate{itemize subsubitem}{\color{mygray}$\blacksquare$}

\setbeamercolor{palette quaternary}{fg=white,bg=mydarkgray}
\setbeamercolor{titlelike}{parent=palette quaternary}

\setbeamercolor{palette quaternary2}{fg=black,bg=mypaleblue}
\setbeamercolor{frametitle}{parent=palette quaternary2}

\setbeamerfont{frametitle}{size=\Large,series=\scshape}
\setbeamerfont{framesubtitle}{size=\normalsize,series=\upshape}





%%%%%%%%%%%%%%%%%%%%%%%%%%%%
% block settings

\setbeamercolor{block title}{bg=red!30,fg=black}

\setbeamercolor*{block title example}{bg=mygreen!40!white,fg=black}

\setbeamercolor*{block body example}{fg= black, bg= mygreen2}


%%%%%%%%%%%%%%%%%%%%%%%%%%%%
% URL settings
\hypersetup{
	colorlinks=true,
	linkcolor=blue,
	filecolor=blue,      
	urlcolor=blue,
}

%%%%%%%%%%%%%%%%%%%%%%%%%%

\renewcommand{\familydefault}{\rmdefault}

\usepackage{amsmath}
\usepackage{mathtools}

\usepackage{subcaption}

\usepackage{qrcode}

\newcommand{\bo}[1] {\mathbf{#1}}
\newcommand{\R}{\mathbb{R}} 
\newcommand{\T}{^\top}     



\newcommand{\mydate}{Spring 2025}

\newcommand{\mygit}{\textcolor{blue}{\href{https://github.com/SergeiSa/Control-Theory-2024}{github.com/SergeiSa/Control-Theory-2024}}}

\newcommand{\myqr}{ \textcolor{black}{\qrcode[height=1.5in]{https://github.com/SergeiSa/Control-Theory-2024}}
}

\newcommand{\myqrframe}{
	\begin{frame}
		\centerline{Lecture slides are available via Github:}
		\bigskip
		\centerline{\mygit}
		\bigskip
		\myqr
	\end{frame}
}


\newcommand{\bref}[2] {\textcolor{blue}{\href{#1}{#2}}}

%%%%%%%%%%%%%%%%%%%%%%%%%%%%
% code settings

\usepackage{listings}
\usepackage{color}
% \definecolor{mygreen}{rgb}{0,0.6,0}
% \definecolor{mygray}{rgb}{0.5,0.5,0.5}
\definecolor{mymauve}{rgb}{0.58,0,0.82}
\lstset{ 
	backgroundcolor=\color{white},   % choose the background color; you must add \usepackage{color} or \usepackage{xcolor}; should come as last argument
	basicstyle=\footnotesize,        % the size of the fonts that are used for the code
	breakatwhitespace=false,         % sets if automatic breaks should only happen at whitespace
	breaklines=true,                 % sets automatic line breaking
	captionpos=b,                    % sets the caption-position to bottom
	commentstyle=\color{mygreen},    % comment style
	deletekeywords={...},            % if you want to delete keywords from the given language
	escapeinside={\%*}{*)},          % if you want to add LaTeX within your code
	extendedchars=true,              % lets you use non-ASCII characters; for 8-bits encodings only, does not work with UTF-8
	firstnumber=0000,                % start line enumeration with line 0000
	frame=single,	                   % adds a frame around the code
	keepspaces=true,                 % keeps spaces in text, useful for keeping indentation of code (possibly needs columns=flexible)
	keywordstyle=\color{blue},       % keyword style
	language=Octave,                 % the language of the code
	morekeywords={*,...},            % if you want to add more keywords to the set
	numbers=left,                    % where to put the line-numbers; possible values are (none, left, right)
	numbersep=5pt,                   % how far the line-numbers are from the code
	numberstyle=\tiny\color{mygray}, % the style that is used for the line-numbers
	rulecolor=\color{black},         % if not set, the frame-color may be changed on line-breaks within not-black text (e.g. comments (green here))
	showspaces=false,                % show spaces everywhere adding particular underscores; it overrides 'showstringspaces'
	showstringspaces=false,          % underline spaces within strings only
	showtabs=false,                  % show tabs within strings adding particular underscores
	stepnumber=2,                    % the step between two line-numbers. If it's 1, each line will be numbered
	stringstyle=\color{mymauve},     % string literal style
	tabsize=2,	                   % sets default tabsize to 2 spaces
	title=\lstname                   % show the filename of files included with \lstinputlisting; also try caption instead of title
}


%%%%%%%%%%%%%%%%%%%%%%%%%%%%
% URL settings
\hypersetup{
	colorlinks=false,
	linkcolor=blue,
	filecolor=blue,      
	urlcolor=blue,
}

%%%%%%%%%%%%%%%%%%%%%%%%%%

%%%%%%%%%%%%%%%%%%%%%%%%%%%%
% tikz settings

\usepackage{tikz}
\tikzset{every picture/.style={line width=0.75pt}}


\title{Lyapunov Theory, Lyapunov equations}
\subtitle{Control Theory, Lecture 12}
\author{by Sergei Savin}
\centering
\date{\mydate}


\begin{document}
\maketitle


\begin{frame}{Content}

\begin{itemize}
\item Lyapunov method: stability criteria
\item Lyapunov method: examples
\item Linear case
\item Discrete case
\item Lyapunov equations
\item Read more
\end{itemize}

\end{frame}





\begin{frame}{Lyapunov method: stability criteria}
% \framesubtitle{Parameter estimation}
\begin{flushleft}

\begin{block}{Asymptotic stability criteria}
Autonomous dynamic system $\dot{\bo{x}} = \bo{f}(\bo{x})$ is assymptotically stable, if there exists a scalar function $V = V(\bo{x}) > 0$, whose time derivative is negative $\dot V(\bo{x}) < 0$, except $V(\bo{0}) = 0$, $\dot V(\bo{0}) = 0$.
\end{block}

\begin{block}{Marginal stability criteria}
$\dot{\bo{x}} = \bo{f}(\bo{x})$ is stable in the sense of Lyapunov, $\exists V(\bo{x}) > 0$, $\dot V(\bo{x}) \leq 0$.
\end{block}

\begin{definition}
Function $V(\bo{x}) > 0$ in this case is called \emph{Lyapunov function}.
\end{definition}


\end{flushleft}
\end{frame}


\begin{frame}{Lyapunov method: Example 1}
%\framesubtitle{Example 1}
\begin{flushleft}

Take dynamical system $\dot{x} = -x$. 

\bigskip

We propose a \emph{Lyapunov function candidate} $V(x) = x^2 \geq 0$. Let's find its derivative:

\begin{equation}
    \dot V(x) = 
    \frac{d}{dt}(  x^2  ) = 
    2x \dot x =
    2x (-x) =
    -x^2 \leq 0
\end{equation}

This satisfies the Lyapunov criteria, so the system is stable. It is in fact asymptotically stable, because $\dot V(x) \neq 0$ if $x \neq 0$.

\end{flushleft}
\end{frame}



\begin{frame}{Lyapunov method: Example 2}
%	\framesubtitle{Example 3}
	\begin{flushleft}
		
		Consider pendulum $\ddot{q} = f(q, \dot{q}) = -\dot{q} - \sin(q)$. 
		
		\bigskip
		
		We propose a \emph{Lyapunov function candidate} $V(q, \dot{q}) = E(q, \dot{q}) = \frac{1}{2} \dot{q}^2 + 1 - \cos(q)\geq 0$, where $E(q, \dot{q})$ is total energy of the system. Let's find its derivative:
		
		
		\begin{align}
			\dot V(q, \dot{q}) = 
			\frac{d}{dt} (\frac{1}{2} \dot{q}^2 + 1 - \cos(q)) =
			\dot{q} \ddot{q} + \sin(q)\dot{q} = \\
			=
			\dot{q} (-\dot{q} - \sin(q)) + \sin(q)\dot{q}
			=
			-\dot{q}^2 \leq 0
		\end{align}
		
		This satisfies the Lyapunov criteria, so the system is stable. It is not proven to be asymptotically stable, because $\dot V(q, \dot{q}) = 0$ for any $q$, as long as $\dot{q} = 0$.
		
	\end{flushleft}
\end{frame}


\begin{frame}{LaSalle's invariance principle, 1}
	% \framesubtitle{Parameter estimation}
	\begin{flushleft}
		
		\begin{block}{LaSalle's invariance principle}
			Autonomous dynamic system $\dot{\bo{x}} = \bo{f}(\bo{x})$ is asymptotically stable, if there exists a scalar function $V = V(\bo{x}) > 0$, whose time derivative is negative $\dot V(\bo{x}) \leq 0$, except $V(\bo{0}) = 0$, where the set $\{\bo{x}: \   \dot V(\bo{x}) = 0  \}$ does not contain non-trivial trajectories.
		\end{block}
	
		\bigskip
		
		A trivial trajectory is $\bo{x}(t) = 0$. Unlike Lyapunov condition, LaSalle's principle allows us to prove asymptotic stability even for systems with $\dot V(\bo{x}) = 0$.
		
		
	\end{flushleft}
\end{frame}




\begin{frame}{LaSalle's invariance principle, 2}
	% \framesubtitle{Parameter estimation}
	\begin{flushleft}
		
		Local version of LaSalle's invariance principle has the following form:
		
		\begin{block}{Local LaSalle's invariance principle}
			Autonomous dynamic system $\dot{\bo{x}} = \bo{f}(\bo{x})$ is asymptotically stable in the neighborhood $\mathcal D$ of the origin, if there exists a scalar function $V = V(\bo{x}) > 0$ (except $V(\bo{0}) = 0$), whose time derivative is non-positive $\dot V(\bo{x}) \leq 0$, where the set $\mathcal M = \{\bo{x}: \   \dot V(\bo{x}) = 0  \} \cap \mathcal D$ does not contain non-trivial trajectories.
		\end{block}
		
		
	\end{flushleft}
\end{frame}



\begin{frame}{LaSalle principle: Example 2}
	%	\framesubtitle{Example 3}
	\begin{flushleft}
		
			In our previous example $\dot V(q, \dot{q}) = 0$ for any $q$, as long as $\dot{q} = 0$. But the set $\{(q, \dot{q}): \   \dot{q} = 0 \}$ contains no trajectories of the system $\ddot{q}  = -\dot{q} - \sin(q)$ other than $q(t) = 0$ in the region $-\frac{\pi}{2} < q < \frac{\pi}{2}$. So, LaSalle principle proves local asymptotic stability.
		
	\end{flushleft}
\end{frame}



\begin{frame}{LaSalle principle: Example 3}
	%\framesubtitle{Example 2}
	\begin{flushleft}
		
		Consider oscillator $\ddot{q} = f(q, \dot{q}) = -\dot{q}$. 
		
		\bigskip
		
		We propose a \emph{Lyapunov function candidate} $V(q, \dot{q}) = T(q, \dot{q}) = \frac{1}{2} \dot{q}^2 \geq 0$, where $T(q, \dot{q})$ is kinetic energy of the system. Let's find its derivative:
		
		\begin{equation}
			\dot V(q, \dot{q}) = 
			\frac{\partial V}{\partial q}       \dot{q} +
			\frac{\partial V}{\partial \dot{q}} f(q, \dot{q}) = 
			\dot{q} (-\dot{q}) = -\dot{q}^2 \leq 0
		\end{equation}
		
		
		This satisfies the Lyapunov criteria, so the system is stable. Note that $\dot V(q, \dot{q}) = 0$ for any $q$ as long as $\dot{q} = 0$. But the set $\{(q, \dot{q}): \   \dot{q} = 0 \}$ contains infinitely many trajectories of the system $\ddot{q} = -\dot{q}$ other than $q(t) = 0$, for example $q(t) = 1$ or $q(t) = -2$. So, LaSalle principle does not prove asymptotic stability in this case.
		
	\end{flushleft}
\end{frame}


%\begin{frame}{LaSalle principle: Example 3}
%	%\framesubtitle{Example 2}
%	\begin{flushleft}
%		
%		In our previous example $\dot V(q, \dot{q}) = 0$ for any $q$ as long as $\dot{q} = 0$. But the set $\{(q, \dot{q}): \   \dot{q} = 0 \}$ contains infinitely many trajectories of the system $\ddot{q} = -\dot{q}$ other than $q(t) = 0$, for example $q(t) = 1$ or $q(t) = -2$. So, LaSalle principle does not prove assymptotic stability in this case.
%		
%	\end{flushleft}
%\end{frame}






\begin{frame}{Linear case, 1}
\begin{flushleft}

As you saw, Lyapunov method allows you to deal with nonlinear systems, as well as linear ones. But for linear systems there are additional properties we can use.

\bigskip

\begin{block}{Observation 1}
For a linear system $\dot{\bo{x}} = \bo{A}\bo{x}$ we can always pick Lyapunov function candidate in the form $V = \bo{x}^\top\bo{S}\bo{x} > 0$, where $\bo{S}$ is a positive definite matrix.
\end{block}

\bigskip

Next slide will shows where this leads us.

\end{flushleft}
\end{frame}


\begin{frame}{Linear case, 2}
\begin{flushleft}

Given $\dot{\bo{x}} = \bo{A}\bo{x}$ and $V = \bo{x}^\top\bo{S}\bo{x} \geq 0$, let's find its derivative:

\begin{equation}
    \dot V(\bo{x}) = \dot{\bo{x}}^\top\bo{S}\bo{x} + 
    \bo{x}^\top\bo{S}\dot{\bo{x}}
\end{equation}

\begin{equation}
    \dot V(\bo{x}) = (\bo{A}\bo{x})^\top\bo{S}\bo{x} + 
    \bo{x}^\top\bo{S}\bo{A}\bo{x} = 
    \bo{x}^\top(\bo{A}^\top\bo{S} + \bo{S}\bo{A})\bo{x}
\end{equation}

Notice that $\dot V(x)$ should be negative for all $\bo{x}$ for the system to be stable, meaning that $\bo{A}^\top\bo{S} + \bo{S}\bo{A}$ should be negative definite. A more strict form of this requirement is \emph{Lyapunov equation}:

\begin{equation}
    \bo{A}^\top\bo{S} + \bo{S}\bo{A} = -\bo{Q}
\end{equation}

where $\bo{Q}$ is a positive-definite matrix.

\end{flushleft}
\end{frame}



\begin{frame}{Discrete case, 1}
\begin{flushleft}

\begin{block}{Asymptotic stability criteria, discrete case}
Given $\bo{x}_{i+1} = \bo{f}(\bo{x}_i)$, if $V(\bo{x}_i) > 0$, and $V(\bo{x}_{i+1}) - V(\bo{x}_i) < 0$, the system is stable.
\end{block}

\bigskip 

Same as before, for linear systems we will be choosing \emph{positive-definite quadratic forms} as Lyapunov function candidates.

\end{flushleft}
\end{frame}



\begin{frame}{Discrete case, 2}
\begin{flushleft}

Consider dynamics $\bo{x}_{i+1} = \bo{A}\bo{x}_i$ and $V = \bo{x}_i^\top\bo{S}\bo{x}_i \geq 0$, let's find $V(\bo{x}_{i+1}) - V(\bo{x}_i)$:

\begin{align}
	V(\bo{x}_{i+1}) - V(\bo{x}_i) &= \bo{x}_{i+1}^\top\bo{S}\bo{x}_{i+1}- 
	\bo{x}_i^\top\bo{S}\bo{x}_i =
	\\
	&= (\bo{A}\bo{x}_i)^\top\bo{S}\bo{A}\bo{x}_i - 
	\bo{x}_i^\top\bo{S}\bo{x}_i
	\\
	&= \bo{x}_i^\top(\bo{A}^\top\bo{S}\bo{A} - \bo{S})\bo{x}_i
\end{align}

Notice that $V(\bo{x}_{i+1}) - V(\bo{x}_i)$ should be negative for all $\bo{x}_i$ for the system to be stable, meaning that $\bo{A}^\top\bo{S}\bo{A} - \bo{S}$ should be negative definite, giving us \emph{Discrete Lyapunov equation}:

\begin{equation}
    \bo{A}^\top\bo{S}\bo{A} - \bo{S} = -\bo{Q}
\end{equation}

where $\bo{Q}$ is a positive-definite matrix.

\end{flushleft}
\end{frame}






\begin{frame}{Lyapunov equations}
% \framesubtitle{Local coordinates}
\begin{flushleft}

In practice, you can easily use Lyapunov equations for stability verification. Python and MATLAB have built-in functionality to solve it:

\begin{itemize}
    \item scipy: \texttt{linalg.solve\_continuous\_lyapunov(A, Q)}
    \item MATLAB: \texttt{lyap(A,Q)}
\end{itemize}

\end{flushleft}
\end{frame}







\begin{frame}{Read more}
% \framesubtitle{Local coordinates}
\begin{flushleft}

\begin{itemize}
    \item \bref{https://folk.uib.no/nmagb/m2142002l3.pdf}{3.9 Liapunov’s direct method}
    \item \bref{https://arxiv.org/abs/1809.05289}{Universita degli studi di Padova Dipartimento di Ingegneria dell'Informazione, Nicoletta Bof, Ruggero Carli, Luca Schenato, Technical Report, Lyapunov Theory for Discrete Time Systems}
\end{itemize}

\end{flushleft}
\end{frame}




\myqrframe

\end{document}
